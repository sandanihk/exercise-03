% Options for packages loaded elsewhere
\PassOptionsToPackage{unicode}{hyperref}
\PassOptionsToPackage{hyphens}{url}
\PassOptionsToPackage{dvipsnames,svgnames,x11names}{xcolor}
%
\documentclass[
  letterpaper,
  DIV=11,
  numbers=noendperiod]{scrartcl}

\usepackage{amsmath,amssymb}
\usepackage{iftex}
\ifPDFTeX
  \usepackage[T1]{fontenc}
  \usepackage[utf8]{inputenc}
  \usepackage{textcomp} % provide euro and other symbols
\else % if luatex or xetex
  \usepackage{unicode-math}
  \defaultfontfeatures{Scale=MatchLowercase}
  \defaultfontfeatures[\rmfamily]{Ligatures=TeX,Scale=1}
\fi
\usepackage{lmodern}
\ifPDFTeX\else  
    % xetex/luatex font selection
\fi
% Use upquote if available, for straight quotes in verbatim environments
\IfFileExists{upquote.sty}{\usepackage{upquote}}{}
\IfFileExists{microtype.sty}{% use microtype if available
  \usepackage[]{microtype}
  \UseMicrotypeSet[protrusion]{basicmath} % disable protrusion for tt fonts
}{}
\makeatletter
\@ifundefined{KOMAClassName}{% if non-KOMA class
  \IfFileExists{parskip.sty}{%
    \usepackage{parskip}
  }{% else
    \setlength{\parindent}{0pt}
    \setlength{\parskip}{6pt plus 2pt minus 1pt}}
}{% if KOMA class
  \KOMAoptions{parskip=half}}
\makeatother
\usepackage{xcolor}
\setlength{\emergencystretch}{3em} % prevent overfull lines
\setcounter{secnumdepth}{-\maxdimen} % remove section numbering
% Make \paragraph and \subparagraph free-standing
\makeatletter
\ifx\paragraph\undefined\else
  \let\oldparagraph\paragraph
  \renewcommand{\paragraph}{
    \@ifstar
      \xxxParagraphStar
      \xxxParagraphNoStar
  }
  \newcommand{\xxxParagraphStar}[1]{\oldparagraph*{#1}\mbox{}}
  \newcommand{\xxxParagraphNoStar}[1]{\oldparagraph{#1}\mbox{}}
\fi
\ifx\subparagraph\undefined\else
  \let\oldsubparagraph\subparagraph
  \renewcommand{\subparagraph}{
    \@ifstar
      \xxxSubParagraphStar
      \xxxSubParagraphNoStar
  }
  \newcommand{\xxxSubParagraphStar}[1]{\oldsubparagraph*{#1}\mbox{}}
  \newcommand{\xxxSubParagraphNoStar}[1]{\oldsubparagraph{#1}\mbox{}}
\fi
\makeatother

\usepackage{color}
\usepackage{fancyvrb}
\newcommand{\VerbBar}{|}
\newcommand{\VERB}{\Verb[commandchars=\\\{\}]}
\DefineVerbatimEnvironment{Highlighting}{Verbatim}{commandchars=\\\{\}}
% Add ',fontsize=\small' for more characters per line
\usepackage{framed}
\definecolor{shadecolor}{RGB}{241,243,245}
\newenvironment{Shaded}{\begin{snugshade}}{\end{snugshade}}
\newcommand{\AlertTok}[1]{\textcolor[rgb]{0.68,0.00,0.00}{#1}}
\newcommand{\AnnotationTok}[1]{\textcolor[rgb]{0.37,0.37,0.37}{#1}}
\newcommand{\AttributeTok}[1]{\textcolor[rgb]{0.40,0.45,0.13}{#1}}
\newcommand{\BaseNTok}[1]{\textcolor[rgb]{0.68,0.00,0.00}{#1}}
\newcommand{\BuiltInTok}[1]{\textcolor[rgb]{0.00,0.23,0.31}{#1}}
\newcommand{\CharTok}[1]{\textcolor[rgb]{0.13,0.47,0.30}{#1}}
\newcommand{\CommentTok}[1]{\textcolor[rgb]{0.37,0.37,0.37}{#1}}
\newcommand{\CommentVarTok}[1]{\textcolor[rgb]{0.37,0.37,0.37}{\textit{#1}}}
\newcommand{\ConstantTok}[1]{\textcolor[rgb]{0.56,0.35,0.01}{#1}}
\newcommand{\ControlFlowTok}[1]{\textcolor[rgb]{0.00,0.23,0.31}{\textbf{#1}}}
\newcommand{\DataTypeTok}[1]{\textcolor[rgb]{0.68,0.00,0.00}{#1}}
\newcommand{\DecValTok}[1]{\textcolor[rgb]{0.68,0.00,0.00}{#1}}
\newcommand{\DocumentationTok}[1]{\textcolor[rgb]{0.37,0.37,0.37}{\textit{#1}}}
\newcommand{\ErrorTok}[1]{\textcolor[rgb]{0.68,0.00,0.00}{#1}}
\newcommand{\ExtensionTok}[1]{\textcolor[rgb]{0.00,0.23,0.31}{#1}}
\newcommand{\FloatTok}[1]{\textcolor[rgb]{0.68,0.00,0.00}{#1}}
\newcommand{\FunctionTok}[1]{\textcolor[rgb]{0.28,0.35,0.67}{#1}}
\newcommand{\ImportTok}[1]{\textcolor[rgb]{0.00,0.46,0.62}{#1}}
\newcommand{\InformationTok}[1]{\textcolor[rgb]{0.37,0.37,0.37}{#1}}
\newcommand{\KeywordTok}[1]{\textcolor[rgb]{0.00,0.23,0.31}{\textbf{#1}}}
\newcommand{\NormalTok}[1]{\textcolor[rgb]{0.00,0.23,0.31}{#1}}
\newcommand{\OperatorTok}[1]{\textcolor[rgb]{0.37,0.37,0.37}{#1}}
\newcommand{\OtherTok}[1]{\textcolor[rgb]{0.00,0.23,0.31}{#1}}
\newcommand{\PreprocessorTok}[1]{\textcolor[rgb]{0.68,0.00,0.00}{#1}}
\newcommand{\RegionMarkerTok}[1]{\textcolor[rgb]{0.00,0.23,0.31}{#1}}
\newcommand{\SpecialCharTok}[1]{\textcolor[rgb]{0.37,0.37,0.37}{#1}}
\newcommand{\SpecialStringTok}[1]{\textcolor[rgb]{0.13,0.47,0.30}{#1}}
\newcommand{\StringTok}[1]{\textcolor[rgb]{0.13,0.47,0.30}{#1}}
\newcommand{\VariableTok}[1]{\textcolor[rgb]{0.07,0.07,0.07}{#1}}
\newcommand{\VerbatimStringTok}[1]{\textcolor[rgb]{0.13,0.47,0.30}{#1}}
\newcommand{\WarningTok}[1]{\textcolor[rgb]{0.37,0.37,0.37}{\textit{#1}}}

\providecommand{\tightlist}{%
  \setlength{\itemsep}{0pt}\setlength{\parskip}{0pt}}\usepackage{longtable,booktabs,array}
\usepackage{calc} % for calculating minipage widths
% Correct order of tables after \paragraph or \subparagraph
\usepackage{etoolbox}
\makeatletter
\patchcmd\longtable{\par}{\if@noskipsec\mbox{}\fi\par}{}{}
\makeatother
% Allow footnotes in longtable head/foot
\IfFileExists{footnotehyper.sty}{\usepackage{footnotehyper}}{\usepackage{footnote}}
\makesavenoteenv{longtable}
\usepackage{graphicx}
\makeatletter
\def\maxwidth{\ifdim\Gin@nat@width>\linewidth\linewidth\else\Gin@nat@width\fi}
\def\maxheight{\ifdim\Gin@nat@height>\textheight\textheight\else\Gin@nat@height\fi}
\makeatother
% Scale images if necessary, so that they will not overflow the page
% margins by default, and it is still possible to overwrite the defaults
% using explicit options in \includegraphics[width, height, ...]{}
\setkeys{Gin}{width=\maxwidth,height=\maxheight,keepaspectratio}
% Set default figure placement to htbp
\makeatletter
\def\fps@figure{htbp}
\makeatother

\KOMAoption{captions}{tableheading}
\makeatletter
\@ifpackageloaded{caption}{}{\usepackage{caption}}
\AtBeginDocument{%
\ifdefined\contentsname
  \renewcommand*\contentsname{Table of contents}
\else
  \newcommand\contentsname{Table of contents}
\fi
\ifdefined\listfigurename
  \renewcommand*\listfigurename{List of Figures}
\else
  \newcommand\listfigurename{List of Figures}
\fi
\ifdefined\listtablename
  \renewcommand*\listtablename{List of Tables}
\else
  \newcommand\listtablename{List of Tables}
\fi
\ifdefined\figurename
  \renewcommand*\figurename{Figure}
\else
  \newcommand\figurename{Figure}
\fi
\ifdefined\tablename
  \renewcommand*\tablename{Table}
\else
  \newcommand\tablename{Table}
\fi
}
\@ifpackageloaded{float}{}{\usepackage{float}}
\floatstyle{ruled}
\@ifundefined{c@chapter}{\newfloat{codelisting}{h}{lop}}{\newfloat{codelisting}{h}{lop}[chapter]}
\floatname{codelisting}{Listing}
\newcommand*\listoflistings{\listof{codelisting}{List of Listings}}
\makeatother
\makeatletter
\makeatother
\makeatletter
\@ifpackageloaded{caption}{}{\usepackage{caption}}
\@ifpackageloaded{subcaption}{}{\usepackage{subcaption}}
\makeatother

\ifLuaTeX
  \usepackage{selnolig}  % disable illegal ligatures
\fi
\usepackage{bookmark}

\IfFileExists{xurl.sty}{\usepackage{xurl}}{} % add URL line breaks if available
\urlstyle{same} % disable monospaced font for URLs
\hypersetup{
  pdftitle={EDA-challenge.qmd},
  pdfauthor={Sandani Kottegoda},
  colorlinks=true,
  linkcolor={blue},
  filecolor={Maroon},
  citecolor={Blue},
  urlcolor={Blue},
  pdfcreator={LaTeX via pandoc}}


\title{EDA-challenge.qmd}
\author{Sandani Kottegoda}
\date{2026-02-07}

\begin{document}
\maketitle


\subsubsection{\texorpdfstring{\textbf{Preliminary
Tasks}}{Preliminary Tasks}}\label{preliminary-tasks}

Loading in and looking at characteristics of the dataset.

\begin{Shaded}
\begin{Highlighting}[]
\FunctionTok{library}\NormalTok{(tidyverse)}
\end{Highlighting}
\end{Shaded}

\begin{verbatim}
-- Attaching core tidyverse packages ------------------------ tidyverse 2.0.0 --
v dplyr     1.1.4     v readr     2.1.6
v forcats   1.0.1     v stringr   1.6.0
v ggplot2   4.0.1     v tibble    3.3.1
v lubridate 1.9.4     v tidyr     1.3.2
v purrr     1.2.1     
-- Conflicts ------------------------------------------ tidyverse_conflicts() --
x dplyr::filter() masks stats::filter()
x dplyr::lag()    masks stats::lag()
i Use the conflicted package (<http://conflicted.r-lib.org/>) to force all conflicts to become errors
\end{verbatim}

\begin{Shaded}
\begin{Highlighting}[]
\NormalTok{f }\OtherTok{\textless{}{-}} \StringTok{"https://raw.githubusercontent.com/difiore/ada{-}datasets/main/data{-}wrangling.csv"}
\NormalTok{d }\OtherTok{\textless{}{-}} \FunctionTok{read\_csv}\NormalTok{(}\AttributeTok{file =}\NormalTok{ f, }\AttributeTok{col\_names =} \ConstantTok{TRUE}\NormalTok{)}
\end{Highlighting}
\end{Shaded}

\begin{verbatim}
Rows: 213 Columns: 23
-- Column specification --------------------------------------------------------
Delimiter: ","
chr  (6): Scientific_Name, Family, Genus, Species, Leaves, Fauna
dbl (17): Brain_Size_Species_Mean, Body_mass_male_mean, Body_mass_female_mea...

i Use `spec()` to retrieve the full column specification for this data.
i Specify the column types or set `show_col_types = FALSE` to quiet this message.
\end{verbatim}

\begin{Shaded}
\begin{Highlighting}[]
\CommentTok{\# Exploring attributes of the dataset}
\FunctionTok{names}\NormalTok{(d)}
\end{Highlighting}
\end{Shaded}

\begin{verbatim}
 [1] "Scientific_Name"         "Family"                 
 [3] "Genus"                   "Species"                
 [5] "Brain_Size_Species_Mean" "Body_mass_male_mean"    
 [7] "Body_mass_female_mean"   "MeanGroupSize"          
 [9] "AdultMales"              "AdultFemale"            
[11] "GR_MidRangeLat_dd"       "Precip_Mean_mm"         
[13] "Temp_Mean_degC"          "HomeRange_km2"          
[15] "DayLength_km"            "Fruit"                  
[17] "Leaves"                  "Fauna"                  
[19] "Canine_Dimorphism"       "Feed"                   
[21] "Move"                    "Rest"                   
[23] "Social"                 
\end{verbatim}

\begin{Shaded}
\begin{Highlighting}[]
\FunctionTok{dim}\NormalTok{(d)}
\end{Highlighting}
\end{Shaded}

\begin{verbatim}
[1] 213  23
\end{verbatim}

\begin{Shaded}
\begin{Highlighting}[]
\FunctionTok{str}\NormalTok{(d)}
\end{Highlighting}
\end{Shaded}

\begin{verbatim}
spc_tbl_ [213 x 23] (S3: spec_tbl_df/tbl_df/tbl/data.frame)
 $ Scientific_Name        : chr [1:213] "Allenopithecus_nigroviridis" "Allocebus_trichotis" "Alouatta_belzebul" "Alouatta_caraya" ...
 $ Family                 : chr [1:213] "Cercopithecidae" "Cercopithecidae" "Atelidae" "Atelidae" ...
 $ Genus                  : chr [1:213] "Allenopithecus" "Allocebus" "Alouatta" "Alouatta" ...
 $ Species                : chr [1:213] "nigroviridis" "trichotis" "belzebul" "caraya" ...
 $ Brain_Size_Species_Mean: num [1:213] 58 NA 52.8 52.6 51.7 ...
 $ Body_mass_male_mean    : num [1:213] 6130 92 7270 6525 5800 ...
 $ Body_mass_female_mean  : num [1:213] 3180 84 5520 4240 4550 5350 6430 5210 1230 NA ...
 $ MeanGroupSize          : num [1:213] NA 1 7 8 6.53 12 6.6 7.1 3.1 3 ...
 $ AdultMales             : num [1:213] NA 1 1 2.3 1.37 ...
 $ AdultFemale            : num [1:213] NA 1 1 3.3 2.2 ...
 $ GR_MidRangeLat_dd      : num [1:213] -0.17 -16.59 -6.8 -20.34 -21.13 ...
 $ Precip_Mean_mm         : num [1:213] 1574 1902 1644 1166 1332 ...
 $ Temp_Mean_degC         : num [1:213] 25.2 20.3 24.9 22.9 19.6 23.7 25.1 25.1 24.6 NA ...
 $ HomeRange_km2          : num [1:213] NA NA NA NA 0.03 0.19 0.3 0.1 0.095 NA ...
 $ DayLength_km           : num [1:213] NA NA NA 0.4 NA 0.32 NA 0.55 NA NA ...
 $ Fruit                  : num [1:213] NA NA 57.3 23.8 5.2 33.1 40.8 40 45 NA ...
 $ Leaves                 : chr [1:213] NA NA "19.1" "67.7" ...
 $ Fauna                  : chr [1:213] NA NA "0" "0" ...
 $ Canine_Dimorphism      : num [1:213] 2.21 NA 1.81 1.54 1.78 ...
 $ Feed                   : num [1:213] NA NA 13.8 15.9 18.3 ...
 $ Move                   : num [1:213] NA NA 18.8 17.6 14.3 ...
 $ Rest                   : num [1:213] NA NA 57.3 61.6 64.4 ...
 $ Social                 : num [1:213] NA NA 10 4.9 3 3.64 3.8 2.5 NA NA ...
 - attr(*, "spec")=
  .. cols(
  ..   Scientific_Name = col_character(),
  ..   Family = col_character(),
  ..   Genus = col_character(),
  ..   Species = col_character(),
  ..   Brain_Size_Species_Mean = col_double(),
  ..   Body_mass_male_mean = col_double(),
  ..   Body_mass_female_mean = col_double(),
  ..   MeanGroupSize = col_double(),
  ..   AdultMales = col_double(),
  ..   AdultFemale = col_double(),
  ..   GR_MidRangeLat_dd = col_double(),
  ..   Precip_Mean_mm = col_double(),
  ..   Temp_Mean_degC = col_double(),
  ..   HomeRange_km2 = col_double(),
  ..   DayLength_km = col_double(),
  ..   Fruit = col_double(),
  ..   Leaves = col_character(),
  ..   Fauna = col_character(),
  ..   Canine_Dimorphism = col_double(),
  ..   Feed = col_double(),
  ..   Move = col_double(),
  ..   Rest = col_double(),
  ..   Social = col_double()
  .. )
 - attr(*, "problems")=<externalptr> 
\end{verbatim}

\begin{Shaded}
\begin{Highlighting}[]
\FunctionTok{head}\NormalTok{(d)}
\end{Highlighting}
\end{Shaded}

\begin{verbatim}
# A tibble: 6 x 23
  Scientific_Name             Family        Genus Species Brain_Size_Species_M~1
  <chr>                       <chr>         <chr> <chr>                    <dbl>
1 Allenopithecus_nigroviridis Cercopitheci~ Alle~ nigrov~                   58.0
2 Allocebus_trichotis         Cercopitheci~ Allo~ tricho~                   NA  
3 Alouatta_belzebul           Atelidae      Alou~ belzeb~                   52.8
4 Alouatta_caraya             Atelidae      Alou~ caraya                    52.6
5 Alouatta_guariba            Atelidae      Alou~ guariba                   51.7
6 Alouatta_palliata           Atelidae      Alou~ pallia~                   49.9
# i abbreviated name: 1: Brain_Size_Species_Mean
# i 18 more variables: Body_mass_male_mean <dbl>, Body_mass_female_mean <dbl>,
#   MeanGroupSize <dbl>, AdultMales <dbl>, AdultFemale <dbl>,
#   GR_MidRangeLat_dd <dbl>, Precip_Mean_mm <dbl>, Temp_Mean_degC <dbl>,
#   HomeRange_km2 <dbl>, DayLength_km <dbl>, Fruit <dbl>, Leaves <chr>,
#   Fauna <chr>, Canine_Dimorphism <dbl>, Feed <dbl>, Move <dbl>, Rest <dbl>,
#   Social <dbl>
\end{verbatim}

\subsubsection{\texorpdfstring{\textbf{Task
\#1}}{Task \#1}}\label{task-1}

\begin{enumerate}
\def\labelenumi{\arabic{enumi}.}
\tightlist
\item
  Create a new variable named BSD (body size dimorphism), the ratio of
  average male to female body mass.
\end{enumerate}

\begin{Shaded}
\begin{Highlighting}[]
\CommentTok{\# adding BSD using $ operator and assigning the ratio of avg. male to female body mass}
\NormalTok{d}\SpecialCharTok{$}\NormalTok{BSD }\OtherTok{\textless{}{-}}\NormalTok{ (d}\SpecialCharTok{$}\NormalTok{Body\_mass\_male\_mean}\SpecialCharTok{/}\NormalTok{d}\SpecialCharTok{$}\NormalTok{Body\_mass\_female\_mean)}
\end{Highlighting}
\end{Shaded}

\subsubsection{\texorpdfstring{\textbf{Task
\#2}}{Task \#2}}\label{task-2}

Create a new variable named sex\_ratio, the ratio of the number of adult
females to males in a group.

\begin{Shaded}
\begin{Highlighting}[]
\CommentTok{\# adding sex\_ratio using $ operator and assigning the ratio of the \# of males to females}
\NormalTok{d}\SpecialCharTok{$}\NormalTok{sex\_ratio }\OtherTok{\textless{}{-}}\NormalTok{ (d}\SpecialCharTok{$}\NormalTok{AdultFemale}\SpecialCharTok{/}\NormalTok{d}\SpecialCharTok{$}\NormalTok{AdultMales)}
\end{Highlighting}
\end{Shaded}

\subsubsection{\texorpdfstring{\textbf{Task
\#3}}{Task \#3}}\label{task-3}

Create a new variable named DI (``defensibility index''), the ratio of
day range length to the diameter of the home range.

\begin{Shaded}
\begin{Highlighting}[]
\CommentTok{\# calculating diameter of home range for each species}
\NormalTok{diameter\_homerange }\OtherTok{\textless{}{-}} \FunctionTok{sqrt}\NormalTok{((d}\SpecialCharTok{$}\NormalTok{HomeRange\_km2}\SpecialCharTok{/}\NormalTok{pi)) }\SpecialCharTok{*} \DecValTok{2} \CommentTok{\# rearranged area equation}


\CommentTok{\# adding DI using $ operator and previous assignment}
\NormalTok{d}\SpecialCharTok{$}\NormalTok{DI }\OtherTok{\textless{}{-}}\NormalTok{ (d}\SpecialCharTok{$}\NormalTok{DayLength\_km}\SpecialCharTok{/}\NormalTok{diameter\_homerange)}
\end{Highlighting}
\end{Shaded}

\subsubsection{\texorpdfstring{\textbf{Task
\#4}}{Task \#4}}\label{task-4}

Plot day range length (y axis) vs.~time spent moving (x axis) for these
primate species overall and by family. Do species that spend more time
moving travel farther overall? How about within any particular primate
family? Should you transform either of these variables? Why or why not?

\textbf{\emph{Note: I filtered out rows that had missing values so there
are no empty graphs when plotting by family.}}

\begin{Shaded}
\begin{Highlighting}[]
\CommentTok{\# filtering out rows that cannot be plotted (have missing values) }
\NormalTok{d\_filtered }\OtherTok{\textless{}{-}}\NormalTok{ d }\SpecialCharTok{|\textgreater{}}
  \FunctionTok{filter}\NormalTok{(}\SpecialCharTok{!}\FunctionTok{is.na}\NormalTok{(Move), }\SpecialCharTok{!}\FunctionTok{is.na}\NormalTok{(DayLength\_km))}


\CommentTok{\# plot for day range length vs. time spent moving overall}
\NormalTok{p\_overall }\OtherTok{\textless{}{-}} \FunctionTok{ggplot}\NormalTok{(}\AttributeTok{data =}\NormalTok{ d\_filtered, }\FunctionTok{aes}\NormalTok{(}\AttributeTok{x =}\NormalTok{ Move, }\AttributeTok{y =} \FunctionTok{log}\NormalTok{(DayLength\_km), }
                                           \AttributeTok{color =} \FunctionTok{factor}\NormalTok{(Family))) }\SpecialCharTok{+} \CommentTok{\# coloring by Family}
  \FunctionTok{xlab}\NormalTok{(}\StringTok{"Time Spent Moving"}\NormalTok{) }\SpecialCharTok{+} \FunctionTok{ylab}\NormalTok{(}\StringTok{"log(Day Range Length (km))"}\NormalTok{) }\SpecialCharTok{+}
  \FunctionTok{geom\_point}\NormalTok{(}\AttributeTok{na.rm =} \ConstantTok{TRUE}\NormalTok{) }\SpecialCharTok{+}
  \FunctionTok{theme}\NormalTok{(}\AttributeTok{legend.position =} \StringTok{"bottom"}\NormalTok{, }\AttributeTok{legend.title =} \FunctionTok{element\_blank}\NormalTok{()) }\CommentTok{\# modifying legend}
\NormalTok{p\_overall }\CommentTok{\# printing output}
\end{Highlighting}
\end{Shaded}

\includegraphics{EDA-challenge_files/figure-pdf/unnamed-chunk-5-1.pdf}

\begin{Shaded}
\begin{Highlighting}[]
\CommentTok{\# plot for day range length vs. time spent moving by family}
\NormalTok{p\_family }\OtherTok{\textless{}{-}} \FunctionTok{ggplot}\NormalTok{(}\AttributeTok{data =}\NormalTok{ d\_filtered, }\FunctionTok{aes}\NormalTok{(}\AttributeTok{x =}\NormalTok{ Move, }\AttributeTok{y =} \FunctionTok{log}\NormalTok{(DayLength\_km), }
                                          \AttributeTok{color =} \FunctionTok{factor}\NormalTok{(Family))) }\SpecialCharTok{+}
  \FunctionTok{xlab}\NormalTok{(}\StringTok{"Time Spent Moving"}\NormalTok{) }\SpecialCharTok{+} \FunctionTok{ylab}\NormalTok{(}\StringTok{"log(Day Range Length (km))"}\NormalTok{) }\SpecialCharTok{+}
  \FunctionTok{geom\_point}\NormalTok{(}\AttributeTok{na.rm =} \ConstantTok{TRUE}\NormalTok{) }\SpecialCharTok{+}
  \FunctionTok{theme}\NormalTok{(}\AttributeTok{legend.position =} \StringTok{"none"}\NormalTok{) }\SpecialCharTok{+} \CommentTok{\# removing legend}
  \FunctionTok{facet\_wrap}\NormalTok{(}\SpecialCharTok{\textasciitilde{}}\NormalTok{Family, }\AttributeTok{ncol =} \DecValTok{4}\NormalTok{)}
\NormalTok{p\_family }\CommentTok{\# printing output}
\end{Highlighting}
\end{Shaded}

\includegraphics{EDA-challenge_files/figure-pdf/unnamed-chunk-5-2.pdf}

\textbf{Response}: Based on the plot, I do not think that species that
spend more time moving travel farther overall because there is no strong
positive relationship between day range length (log-transformed) and
move time. Although there are species that move more and travel farther,
the data is quite variable. If time spent moving predicted day range
length, you would expect to see a clear positive trend, which is not
evident.

Within the Hylobatidae family, there appears to be a positive
association between time spent moving and day range length; however,
there are only two observations, so it is hard to conclude that move
time strongly predicts distance traveled. The other primate families
exhibit high within-family variability and show no consistent trend
between the two variables, even in larger sample sizes.

Lastly, the day range length variable should be log-transformed, because
prior to transformation, the data were right-skewed with most species
exhibiting short travel distances and a few with much larger values. The
transformation allowed for a more symmetric distribution and
interpretation of the data as well. The move time variable does not need
to be log-transformed because it does not span orders of magnitude like
distance does and there is no strong skew.

\subsubsection{\texorpdfstring{\textbf{Task
\#5}}{Task \#5}}\label{task-5}

Plot day range length (y axis) vs.~group size (x axis), overall and by
family. Do species that live in larger groups travel farther, overall,
in a day? How about within any particular primate family? Should you
transform either of these variables?

\textbf{\emph{Note: Again, I filtered out rows that had missing values
so there are no empty graphs when plotting by family.}}

\begin{Shaded}
\begin{Highlighting}[]
\CommentTok{\# Filtering out rows that cannot be plotted (have missing values) }
\NormalTok{d\_clean }\OtherTok{\textless{}{-}}\NormalTok{ d }\SpecialCharTok{|\textgreater{}}
  \FunctionTok{filter}\NormalTok{(}\SpecialCharTok{!}\FunctionTok{is.na}\NormalTok{(MeanGroupSize), }\SpecialCharTok{!}\FunctionTok{is.na}\NormalTok{(DayLength\_km))}

\CommentTok{\# plot for day range length vs. avg. group size overall}
\NormalTok{p\_overall }\OtherTok{\textless{}{-}} \FunctionTok{ggplot}\NormalTok{(}\AttributeTok{data =}\NormalTok{ d\_clean, }\FunctionTok{aes}\NormalTok{(}\AttributeTok{x =}\NormalTok{ MeanGroupSize, }\AttributeTok{y =} \FunctionTok{log}\NormalTok{(DayLength\_km), }
                                        \AttributeTok{color =} \FunctionTok{factor}\NormalTok{(Family))) }\SpecialCharTok{+} \CommentTok{\# coloring by family}
  \FunctionTok{xlab}\NormalTok{(}\StringTok{"Average Group Size"}\NormalTok{) }\SpecialCharTok{+} \FunctionTok{ylab}\NormalTok{(}\StringTok{"log(Day Range Length (km))"}\NormalTok{) }\SpecialCharTok{+}
  \FunctionTok{geom\_point}\NormalTok{(}\AttributeTok{na.rm =} \ConstantTok{TRUE}\NormalTok{) }\SpecialCharTok{+}
  \FunctionTok{theme}\NormalTok{(}\AttributeTok{legend.position =} \StringTok{"bottom"}\NormalTok{, }\AttributeTok{legend.title =} \FunctionTok{element\_blank}\NormalTok{()) }\CommentTok{\# modifying legend}
\NormalTok{p\_overall }\CommentTok{\# printing output}
\end{Highlighting}
\end{Shaded}

\includegraphics{EDA-challenge_files/figure-pdf/unnamed-chunk-6-1.pdf}

\begin{Shaded}
\begin{Highlighting}[]
\CommentTok{\# plot for day range length vs. avg. group size by family}
\NormalTok{p\_family }\OtherTok{\textless{}{-}} \FunctionTok{ggplot}\NormalTok{(}\AttributeTok{data =}\NormalTok{ d\_clean, }\FunctionTok{aes}\NormalTok{(}\AttributeTok{x =}\NormalTok{ MeanGroupSize, }\AttributeTok{y =} \FunctionTok{log}\NormalTok{(DayLength\_km), }
                                       \AttributeTok{color =} \FunctionTok{factor}\NormalTok{(Family))) }\SpecialCharTok{+} 
  \FunctionTok{xlab}\NormalTok{(}\StringTok{"Average Group Size"}\NormalTok{) }\SpecialCharTok{+} \FunctionTok{ylab}\NormalTok{(}\StringTok{"log(Day Range Length (km))"}\NormalTok{) }\SpecialCharTok{+}
  \FunctionTok{geom\_point}\NormalTok{(}\AttributeTok{na.rm =} \ConstantTok{TRUE}\NormalTok{) }\SpecialCharTok{+}
  \FunctionTok{theme}\NormalTok{(}\AttributeTok{legend.position =} \StringTok{"none"}\NormalTok{) }\SpecialCharTok{+} \CommentTok{\# removing legend}
  \FunctionTok{facet\_wrap}\NormalTok{(}\SpecialCharTok{\textasciitilde{}}\NormalTok{Family, }\AttributeTok{ncol =} \DecValTok{4}\NormalTok{)}
\NormalTok{p\_family }\CommentTok{\#printing output}
\end{Highlighting}
\end{Shaded}

\includegraphics{EDA-challenge_files/figure-pdf/unnamed-chunk-6-2.pdf}

\textbf{Response}: Here, there is a weak positive correlation between
average group size and day range length (log-transformed), such that
species in larger groups tend to travel farther per day. However, there
is substantial variability, so group size alone does not strongly
predict distance traveled.

There appears to be a weak positive trend between group size and
distance traveled within the Pitheciidae and Cebidae families, with
species in larger groups tending to travel farther. Cercopithecidae also
shows an overall positive pattern, though, with variability across
species. Hominidae shows a similar trend, but conclusions are limited
due to fewer observations.

Again, the day range length was log-transformed since the data in this
variable span orders magnitudes and are skewed before transformation.
This helps with data visualization and interpretation of this variable.
The average group size has a moderate range that is already
interpretable without transformation and has no severe skew, so it was
not log-transformed.

\subsubsection{\texorpdfstring{\textbf{Task
\#6}}{Task \#6}}\label{task-6}

Plot canine size dimorphism (y axis) vs.~body size dimorphism (x axis)
overall and by family. Do taxa with greater size dimorphism also show
greater canine dimorphism?

\begin{Shaded}
\begin{Highlighting}[]
\CommentTok{\# Filtering out rows that cannot be plotted (have missing values) }
\NormalTok{d\_pure }\OtherTok{\textless{}{-}}\NormalTok{ d }\SpecialCharTok{|\textgreater{}}
  \FunctionTok{filter}\NormalTok{(}\SpecialCharTok{!}\FunctionTok{is.na}\NormalTok{(BSD), }\SpecialCharTok{!}\FunctionTok{is.na}\NormalTok{(Canine\_Dimorphism))}


\CommentTok{\# plot for canine size dimorphism vs. BSD overall}
\NormalTok{p\_overall }\OtherTok{\textless{}{-}} \FunctionTok{ggplot}\NormalTok{(}\AttributeTok{data =}\NormalTok{ d\_pure, }\FunctionTok{aes}\NormalTok{(}\AttributeTok{x =} \FunctionTok{log}\NormalTok{(BSD), }\AttributeTok{y =} \FunctionTok{log}\NormalTok{(Canine\_Dimorphism), }
                                       \AttributeTok{color =} \FunctionTok{factor}\NormalTok{(Family))) }\SpecialCharTok{+} \CommentTok{\# coloring by family}
  \FunctionTok{xlab}\NormalTok{(}\StringTok{"log(Body Size Dimorphism)"}\NormalTok{) }\SpecialCharTok{+} \FunctionTok{ylab}\NormalTok{(}\StringTok{"log(Canine Size Dimorphism)"}\NormalTok{) }\SpecialCharTok{+}
  \FunctionTok{geom\_point}\NormalTok{(}\AttributeTok{na.rm =} \ConstantTok{TRUE}\NormalTok{) }\SpecialCharTok{+}
  \FunctionTok{theme}\NormalTok{(}\AttributeTok{legend.position =} \StringTok{"bottom"}\NormalTok{, }\AttributeTok{legend.title =} \FunctionTok{element\_blank}\NormalTok{()) }\CommentTok{\# editing legend}
\NormalTok{p\_overall }\CommentTok{\# printing output}
\end{Highlighting}
\end{Shaded}

\includegraphics{EDA-challenge_files/figure-pdf/unnamed-chunk-7-1.pdf}

\begin{Shaded}
\begin{Highlighting}[]
\CommentTok{\# plot for canine size dimorphism vs. BSD by family}
\NormalTok{p\_family }\OtherTok{\textless{}{-}} \FunctionTok{ggplot}\NormalTok{(}\AttributeTok{data =}\NormalTok{ d\_pure, }\FunctionTok{aes}\NormalTok{(}\AttributeTok{x =} \FunctionTok{log}\NormalTok{(BSD), }\AttributeTok{y =} \FunctionTok{log}\NormalTok{(Canine\_Dimorphism), }
                                      \AttributeTok{color =} \FunctionTok{factor}\NormalTok{(Family))) }\SpecialCharTok{+} 
  \FunctionTok{xlab}\NormalTok{(}\StringTok{"log(Body Size Dimorphism)"}\NormalTok{) }\SpecialCharTok{+} \FunctionTok{ylab}\NormalTok{(}\StringTok{"log(Canine Size Dimorphism)"}\NormalTok{) }\SpecialCharTok{+}
  \FunctionTok{geom\_point}\NormalTok{(}\AttributeTok{na.rm =} \ConstantTok{TRUE}\NormalTok{) }\SpecialCharTok{+}
  \FunctionTok{theme}\NormalTok{(}\AttributeTok{legend.position =} \StringTok{"none"}\NormalTok{) }\SpecialCharTok{+} \CommentTok{\# removing legend}
  \FunctionTok{facet\_wrap}\NormalTok{(}\SpecialCharTok{\textasciitilde{}}\NormalTok{Family, }\AttributeTok{ncol =} \DecValTok{4}\NormalTok{)}
\NormalTok{p\_family }\CommentTok{\# printing output}
\end{Highlighting}
\end{Shaded}

\includegraphics{EDA-challenge_files/figure-pdf/unnamed-chunk-7-2.pdf}

\textbf{Response}: I think that taxa with greater size dimorphism
generally show greater canine dimorphism, as the plot shows a positive
correlation between the two log-transformed variables. However, this
relationship is more pronounced in certain families like Cercopithecidae
and Hominidae.

By plotting by family, you can see that Cebidae similarly shows a
positive trend; though, their size dimorphism is lower compared to the
other two primate families that were previously mentioned.

Both variables were log-transformed, as the data were skewed are hard to
interpret before transformation.

\subsubsection{\texorpdfstring{\textbf{Task
\#7}}{Task \#7}}\label{task-7}

Create a new variable named diet\_strategy that is ``frugivore'' if
fruits make up \textgreater50\% of the diet, ``folivore'' if leaves make
up \textgreater50\% of the diet, and ``omnivore'' if diet data are
available but neither of these is true. Create boxplots of group size
for species with different dietary strategies. Do frugivores live in
larger groups than folivores?

Here, I am creating diet\_strategy using mutate() as part of \{dplyr\}.
This allows me to add a new column to d.~In the new variable, I assigned
whether the observations were frugivores, folivores, or omnivores (given
that their data was avaliable) based on the criteria above by using
multiple if..else.. statements and passing them through case\_when().

\begin{Shaded}
\begin{Highlighting}[]
\CommentTok{\# Creating diet\_strategy and using conditional statements}
\NormalTok{d }\OtherTok{\textless{}{-}} \FunctionTok{mutate}\NormalTok{(d, }\AttributeTok{diet\_strategy =} \FunctionTok{case\_when}\NormalTok{(Fruit }\SpecialCharTok{\textgreater{}=} \FloatTok{50.0} \SpecialCharTok{\textasciitilde{}} \StringTok{"frugivore"}\NormalTok{,}
\NormalTok{                                   Leaves }\SpecialCharTok{\textgreater{}=} \FloatTok{50.0} \SpecialCharTok{\textasciitilde{}} \StringTok{"folivore"}\NormalTok{,}
\NormalTok{                                   Fruit }\SpecialCharTok{\textless{}} \FloatTok{50.0} \SpecialCharTok{\&}\NormalTok{ Leaves }\SpecialCharTok{\textless{}} \FloatTok{50.0} \SpecialCharTok{\textasciitilde{}} \StringTok{"omnivore"}\NormalTok{, }\CommentTok{\# to combine conditions}
                                   \ConstantTok{TRUE} \SpecialCharTok{\textasciitilde{}} \ConstantTok{NA}\NormalTok{))}


\CommentTok{\# omitting NA category}
\NormalTok{d\_polished }\OtherTok{\textless{}{-}}\NormalTok{ d}\SpecialCharTok{|\textgreater{}}
  \FunctionTok{filter}\NormalTok{(}\SpecialCharTok{!}\FunctionTok{is.na}\NormalTok{(diet\_strategy)) }
  
\CommentTok{\# plotting group size with dietary strategies}
\FunctionTok{boxplot}\NormalTok{(MeanGroupSize }\SpecialCharTok{\textasciitilde{}}\NormalTok{ diet\_strategy,}
        \AttributeTok{data =}\NormalTok{ d\_polished,}
        \AttributeTok{xlab =} \StringTok{"Diet Strategy"}\NormalTok{,}
        \AttributeTok{ylab =} \StringTok{"Average Group Size"}\NormalTok{,}
        \AttributeTok{na.rm =} \ConstantTok{TRUE}\NormalTok{)}
\end{Highlighting}
\end{Shaded}

\includegraphics{EDA-challenge_files/figure-pdf/unnamed-chunk-8-1.pdf}

\textbf{Response}: Based on the boxplots, frugivores generally do not
live in larger groups than folivores. Although frugivores have more
outliers/species of larger group size, their median group size is
slightly lower than the folivore median. Additionally, the interquartile
range between frugivores and folivores largely overlap and suggest a
similar group size distribution between the two groups.

\subsubsection{\texorpdfstring{\textbf{Task
\#8}}{Task \#8}}\label{task-8}

In one line of code: add the variable, Binomial to the data frame (a
concatenation of the Genus and Species variables), trim the data frame
to only include the variables Binomial, Family,
Brain\_size\_species\_mean, and Body\_mass\_male\_mean, group these
variables by Family, calculate the average value for
Brain\_Size\_Species\_Mean and Body\_mass\_male\_mean per Family
(remember, you may need to specify na.rm = TRUE), arrange by increasing
average brain size, and print the output to the console.

\textbf{\emph{Note: I have added parentheses around the assignment to
print the output to the console. Removing the assignment could have been
done as well.}}

\begin{Shaded}
\begin{Highlighting}[]
\NormalTok{(s }\OtherTok{\textless{}{-}}\NormalTok{ d }\SpecialCharTok{|\textgreater{}}
  \CommentTok{\# adding a variable using paste() and specifying the separator}
  \FunctionTok{mutate}\NormalTok{(}\AttributeTok{Binomial =} \FunctionTok{paste}\NormalTok{(Genus, Species, }\AttributeTok{sep =} \StringTok{" "}\NormalTok{)) }\SpecialCharTok{|\textgreater{}} 
  \CommentTok{\# selecting specific variables}
  \FunctionTok{select}\NormalTok{(Binomial, Family, Brain\_Size\_Species\_Mean, Body\_mass\_male\_mean) }\SpecialCharTok{|\textgreater{}}
  \FunctionTok{group\_by}\NormalTok{(Family) }\SpecialCharTok{|\textgreater{}} \CommentTok{\# grouping by family}
  \CommentTok{\# calculating mean brain size and male mass per family}
  \FunctionTok{summarize}\NormalTok{(}\AttributeTok{avg\_brain\_size =} \FunctionTok{mean}\NormalTok{(Brain\_Size\_Species\_Mean, }\AttributeTok{na.rm =} \ConstantTok{TRUE}\NormalTok{),}
            \AttributeTok{avg\_mass\_M =} \FunctionTok{mean}\NormalTok{(Body\_mass\_male\_mean, }\AttributeTok{na.rm =} \ConstantTok{TRUE}\NormalTok{)) }\SpecialCharTok{|\textgreater{}}
   \CommentTok{\# arranging brain size by increasing }
  \FunctionTok{arrange}\NormalTok{(avg\_brain\_size))}
\end{Highlighting}
\end{Shaded}

\begin{verbatim}
# A tibble: 14 x 3
   Family          avg_brain_size avg_mass_M
   <chr>                    <dbl>      <dbl>
 1 Tarsiidae                 3.26       131 
 2 Cheirogalidae             4.04       193.
 3 Galagidae                 5.96       395.
 4 Lepilemuridae             7.27       792 
 5 Lorisidae                 8.67       512.
 6 Lemuridae                23.1       2077.
 7 Cebidae                  23.9       1012.
 8 Indriidae                27.3       3638.
 9 Daubentonidae            44.8       2620 
10 Pitheciidae              56.3       1955.
11 Atelidae                 80.6       7895.
12 Cercopithecidae          85.4       9543.
13 Hylobatidae             101.        6926.
14 Hominidae               410.       98681.
\end{verbatim}




\end{document}
